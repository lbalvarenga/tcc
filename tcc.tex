%----------------------------------------------------------------
%% Template adaptado para preparação de trabalho de conclusão de 
%% curso   (TCC)   do   Departamento   de   Economia   da 
%% Universidade      Estadual     de      Londrina       (UEL)
%
%% Autores: 
%% Prof Ph.D. Joanna G. Alexopoulos
%% Prof. Dr. Marcelo S. Bego 
%%
%% Modificações:
%% Lucas Alvarenga
%
%% Data da última atualização: 25/08/2024
%----------------------------------------------------------------
%%
%% Customizações do abnTeX2 (http://abnTeX2.googlecode.com) 
%%
%% Este trabalho pode ser distribuído e/ou modificado sob as
%% condições da Licença Pública do LaTeX Project, versão 1.3.
%% A versão desta licença pode ser encontrada em:
%%   http://www.latex-project.org/lppl.txt
%% e faz parte da versão LaTeX 01/12/2005.
%%
%% Este trabalho tem como status de manutenção LPPL como "maintained".
%% 
%% Os atuais responsáveis por esse trabalho são: 
%% Prof Ph.D. Joanna G. Alexopoulos
%% Prof. Dr. Marcelo S. Bego 
%%
%% Mais informações sobre abnTeX2 estão disponíveis em: 
%%  https://github.com/abntex/abntex2
%%
%----------------------------------------------------------------
%% Sobre a classe abntex2.cls:
%% abntex2.cls, v-1.9.5 laurocesar
%% Copyright 2012-2015 by abnTeX2 group at https://www.abntex.net.br/ 
%%
%----------------------------------------------------------------

\documentclass[12pt,oneside,a4paper,chapter=TITLE,english,brazil,sumario=abnt-6027-2012]{abntex2}

% Modo escuro %%%%
% \usepackage{xcolor}
% \pagecolor[rgb]{0.1,0.1,0.15} %black
% \color[rgb]{0.5,0.5,0.5} %grey
% %%%%%%%%%%%%%%%% 

% %%%%%%%%%%%%%%%%
% PACOTES
% %%%%%%%%%%%%%%%%

\usepackage{lmodern}			
\usepackage[T1]{fontenc}		
\usepackage[utf8]{inputenc}		
\usepackage{indentfirst}		
\usepackage[dvipsnames]{xcolor}				
\usepackage{graphicx}			
\usepackage{microtype} 
\usepackage{multicol}
\usepackage{multirow}
\usepackage{float}	
\usepackage{lipsum}				
\usepackage{tikz}
\usepackage{ragged2e} 
\usepackage[brazilian,hyperpageref]{backref}	
\usepackage[alf]{abntex2cite}
\usepackage{blindtext} 
\usepackage{mathptmx}
\usepackage[labelfont=bf]{caption}
\captionsetup{labelfont=bf}
\usepackage[singlelinecheck=false]{caption}
\usepackage{amsmath}
\usepackage[table]{xcolor} % Para colorir tabelas
\usepackage{zref-totpages}
\usepackage{titlesec}


% %%%%%%%%%%%%%%%%
%MARGENS
%As margens são configuradas conforme a NBR 14724:2011
%As margens devem ser: para o anverso, esquerda e superior de 3 cm e direita e inferior de 2 cm; para o verso, direita e superior de 3 cm e esquerda e inferior de 2 cm.

% %%%%%%%%%%%%%%%%
%FONTES 
% %%%%%%%%%%%%%%%%

% %%%%%%%%%%%%%%%%
%TIPO DE FONTE
\renewcommand{\sfdefault}{\rmdefault} % Fonte: Times New Roman

% %%%%%%%%%%%%%%%%
% TAMANHO E ESTILO DAS FONTES DOS TÍTULOS E SUBSEÇÕES 
\renewcommand{\ABNTEXchapterfontsize}{\large \bfseries}%14pt

\renewcommand{\ABNTEXsectionfont}{\bfseries}%Subseções em negrito.

\definecolor{verdeUEL}{RGB}{10,128,20}


\setlength\afterchapskip{18pt}

% %%%%%%%%%%%%%%%%
% ESPAÇAMENTOS
% %%%%%%%%%%%%%%%%

% TAMANHO DO PARÁGRAFO 
\setlength{\parindent}{1.27cm} %Primeira linha

%ESPAÇO ENTRE LINHAS
%O padrão é definido como \OnehalfSpacing (1.5)


% %%%%%%%%%%%%%%%%
%INFORMAÇÕES DA CAPA, FOLHA DE ROSTO E FOLHA DE APROVAÇÃO 
% %%%%%%%%%%%%%%%%

% %%%%%%%%%%%%%%%%
%NOME DO TRABALHO 
\titulo{\bfseries ANÁLISE DA DINÂMICA INFLACIONÁRIA SOB O REGIME DE METAS DE INFLAÇÃO NO BRASIL ENTRE 2003 e 2023}


% %%%%%%%%%%%%%%%%
%NOME DO AUTOR 
\autor{LUCAS BONI DOS ANJOS AMARAL ALVARENGA}


% %%%%%%%%%%%%%%%%
%NOME DO ORIENTADOR
\orientador{Carlos Eduardo Caldarelli}


% %%%%%%%%%%%%%%%%
\local{Londrina}
\data{\the\year}
\tipotrabalho{monografia}
\instituicao{Universidade Estadual de Londrina (UEL)}
\preambulo{Trabalho de Conclusão de Curso apresentado ao Departamento de Economia da Universidade  Estadual de Londrina.}
% %%%%%%%%%%%%%%%%


% %%%%%%%%%%%%%%%%
% CAPA UEL
\renewcommand{\imprimircapa}{
	\begin{capa}
		\center
		\begin{figure}
		\includegraphics[height=3.65cm,width=15.5cm]{fig/Logo_UEL_new.png}
		\end{figure}

		% \begin{tikzpicture}
		% \fill[verdeUEL] (17.1,1.25) rectangle (1,1);
		% \end{tikzpicture}
		% \large 
		
		{\bfseries CENTRO DE ESTUDOS SOCIAIS APLICADOS\\
		DEPARTAMENTO DE ECONOMIA\\
		CURSO DE CIÊNCIAS ECONÔMICAS\\}
		
		
		\vspace{5cm}
		
		{\ABNTEXchapterfont\textsc{\Large\imprimirtitulo}}
		
		\vfill
		
		\begin{flushright}
			\imprimirautor
			\vspace{-0.5cm}
			
		\end{flushright}
		
		\begin{tikzpicture}
		\fill[verdeUEL] (17.1,1.125) rectangle (1,1);
		\end{tikzpicture}
		
		{\imprimirlocal}
		
		{\imprimirdata}
		
	\end{capa}
}


\emergencystretch 3em
\hyphenpenalty 10000
\exhyphenpenalty 10000

% %%%%%%%%%%%%%%%%
% FOLHA DE ROSTO
\makeatletter

\renewcommand{\folhaderostocontent}{
	\begin{center}
      \large 	
	
	  {\ABNTEXchapterfont\textsc{\Large \imprimirautor}}
		
		\vspace{6cm}
		
		{\ABNTEXchapterfont\textsc{\Large \imprimirtitulo}}
		
		\vspace{2cm}
		
		\begin{flushright}
			\begin{minipage}{8cm}
				\SingleSpacing

				Trabalho de Conclusão de Curso apresentado à Universidade Estadual de Londrina - UEL, como requisito parcial para a obtenção do título de Bacharel em Ciências Econômicas.
				\break
				
				Orientador: Prof. Dr. \imprimirorientador
			\end{minipage}%
		\end{flushright}
		
		\vfill
		
		
		{\large\imprimirlocal}
		
		{\large\imprimirdata}
	\end{center}
}
\makeatother


% %%%%%%%%%%%%%%%%
% FOLHA DE APROVAÇÃO 
\makeatother

\newcommand{\folhaDeaprovacao}{
	\begin{center}
		
		\begin{folhadeaprovacao} 
			\begin{center} 
				
					
				
			{\ABNTEXchapterfont\textsc{\large \imprimirautor}}
				
				\vspace{2cm}
				
			{\ABNTEXchapterfont\textsc{\large \imprimirtitulo}}
				
				\vspace{1cm}		
				\begin{flushright}
					\begin{minipage}{10cm}
						\SingleSpacing
						Trabalho de Conclusão de Curso apresentado à Universidade Estadual de Londrina - UEL, como requisito parcial para a obtenção do título de Bacharel em Ciências Econômicas.
					\end{minipage}%
				\end{flushright}	
				
				\vspace{2cm}
				
				\begin{flushright}
					\begin{minipage}{10cm}		
						\centering
					{\bfseries COMISSÃO EXAMINADORA}
					\end{minipage}
				\end{flushright}	
				
				\vspace{1cm}
				
				\begin{flushright}
					\begin{minipage}{10cm}	
						\centering
						\hrule \hspace{0.2cm}
						
						Orientador(a): Prof(a). \imprimirorientador  
						
						\imprimirinstituicao
					\end{minipage}%
				\end{flushright}
				
				\vspace{1cm}
				
				\begin{flushright}
					\begin{minipage}{10cm}	
						\centering
						\hrule \hspace{0.2cm}
						
						Prof(a). NOME BANCA 1
						
						\imprimirinstituicao
					\end{minipage}%
				\end{flushright}		
				
				
				\vspace{1cm}
				
				\begin{flushright}
					\begin{minipage}{10cm}	
						\centering
						\hrule \hspace{0.2cm}
						
						Prof(a). NOME BANCA 2 
						
						\imprimirinstituicao
					\end{minipage}%
				\end{flushright}
				
				\vfill 
				
				\begin{flushright}
					\begin{minipage}{6cm}	
				\centering
				Londrina, 27 de Agosto de \imprimirdata.
					\end{minipage}
			\end{flushright}
		
			\end{center} 
		\end{folhadeaprovacao}
	\end{center}
}
\makeatother


% %%%%%%%%%%%%%%%%
%REGULAMENTO DO TRABALHO DE CONCLUSÃO DO CURSO DE ECONOMIA (DISCIPLINAS MONOGRAFIA I E II – 6TCC404 E 6TCC405)
% %%%%%%%%%%%%%%%%

%ART.16 A estrutura da monografia compõe-se de:

%I. Capa;
%II. Folha de rosto;
%III. Elementos pré-textuais;
%IV. Resumo e “Abstract” (ou versão em Espanhol ou ainda em Francês):
%V. Sumário;
%VI. Introdução;
%VII. Desenvolvimento, contendo Referencial Teórico ou Econométrico ou Revisão de Literatura ou Revisão Bibliográfica ou Modelo Teórico; e nos casos de estudos quantitativos Metodologia e Fonte e tratamento de dados;
%VIII. Resultados e discussão;
%IX. Considerações finais ou Conclusão;
%X. Referências;
%XI. Anexos e Apêndices, quando for o caso
%%%%%%%%%%%%%%%%%%%%%%%%%%%%



% INÍCIO DO DOCUMENTO
\begin{document}

% Retira espaço extra obsoleto entre as frases.
\frenchspacing 


% %%%%%%%%%%%%%%%%
% CAPA
\imprimircapa

% %%%%%%%%%%%%%%%%
%FOLHA DE ROSTO
\folhaderostocontent


% %%%%%%%%%%%%%%%%
%FIXA CATALOGRÁFICA 
%Incluir fica catalográfica
%\pagebreak


% %%%%%%%%%%%%%%%%
%FOLHA DE APROVAÇÃO 
\folhaDeaprovacao


% %%%%%%%%%%%%%%%%
%DEDICATÓRIA
\begin{dedicatoria}
	\vspace*{\fill}
	\noindent
	Dedico esse trabalho aos meus pais, Bruno e Silvia, por todo amor, apoio e incentivo, que foram fundamentais para a realização de mais uma conquista. Sua dedicação e sacrifícios me proporcionaram a oportunidade de realizar meus sonhos.
	\vspace*{\fill}
\end{dedicatoria}


% %%%%%%%%%%%%%%%%
%AGRADECIMENTOS
\begin{agradecimentos}
	\noindent
	Agradeço ao meu orientador, Prof. Dr. Carlos Eduardo Caldarelli, por toda ajuda, conhecimento e experiência que compartilhou comigo durante a confecção desse trabalho.
	Agradeço a todos meus professores e a todos os meus colegas de curso, que contribuíram com conhecimentos, experiências e incentivos, tornando minha jornada mais rica e gratificante. Aos amigos, que estiveram presentes nos momentos de lazer e de estudo, proporcionando equilíbrio e motivação durante o percurso.	À Universidade Estadual de Londrina por oferecer um ambiente de aprendizado e crescimento tão acolhedor. A todos aqueles que, direta ou indiretamente, colaboraram para a conclusão deste trabalho.
\end{agradecimentos}


% %%%%%%%%%%%%%%%%
%RESUMO

\setlength{\absparsep}{18pt} % ajusta o espaçamento dos parágrafos do resumo
\begin{resumo}

	\noindent
	ALVARENGA, Lucas. {\bfseries Análise da Dinâmica Inflacionária Sob o Regime de Metas de Inflação no Brasil Entre 2003 e 2023}. \imprimirdata. \ztotpages \, f. Monografia (Graduação em Ciências Econômicas). Centro de Estudos Sociais Aplicados, Universidade Estadual de Londrina, Londrina, 2024.
	
	Este trabalho tem por objetivo verificar o impacto da inflação, principalmente sob a ótica da oferta e da demanda sobre a economia brasileira e avaliar os efeitos exercidos por estas sobre a eficácia do regime de metas de inflação (RMI) no Brasil. Será examinado se as respostas do Banco Central são eficientes e sob quais óticas a política monetária deve ser definida a fim de atender os objetivos propostos pelo RMI. Por meio deste trabalho, busca-se atingir um melhor entendimento dos mecanismos que regem a dinâmica inflacionária no Brasil e como é moldada a resposta à inflação pela autoridade monetária brasileira. Para isso, são utilizados modelos econométricos que conseguem capturar relações de longo prazo entre as variáveis escolhidas, que são a taxa de inflação, a taxa de juros, o produto interno bruto (PIB) real do Brasil e... 
	
	TODO: all variables
	
	\textbf{Palavras-chave}: inflação; oferta; demanda; econometria; política monetária.
\end{resumo}


% %%%%%%%%%%%%%%%%
%ABSTRACT
% \noindent
% ALVARENGA, Lucas. {\bfseries Análise da Dinâmica Inflacionária Sob o Regime de Metas de Inflação no Brasil Entre 2003 e 2023}, \imprimirdata. <FOLHAS> f. Monografia (Curso de Ciências Econômicas). Centro de Estudos Sociais Aplicados, Universidade Estadual de Londrina, Londrina, 2024.

% \begin{resumo}[Abstract]
% 	\begin{otherlanguage*}{english}

% 		Write your abstract here... 

% 		\vspace{\onelineskip}

% 		\noindent 
% 		\textbf{Keywords}: 
% 	\end{otherlanguage*}
% \end{resumo}
% \pagebreak


% %%%%%%%%%%%%%%%%
%LISTA DE FIGURAS
\pdfbookmark[0]{\listfigurename}{lof}
\listoffigures*
\cleardoublepage


% %%%%%%%%%%%%%%%%
%LISTA DE TABELAS
\pdfbookmark[0]{\listtablename}{lot}
\listoftables*
\cleardoublepage


% %%%%%%%%%%%%%%%%
%LISTA DE ABREVIATURAS
\begin{siglas}
	\item[UEL] Universidade Estadual de Londrina. 
	\item[ABNT] Associação Brasileira de Normas Técnicas.
	\item[RMI] Regime de metas de inflação
	\item[SELIC] Sistema Especial de Liquidação e Custódia
	\item[IPCA] Índice Nacional de Preços ao Consumidor Amplo
	\item[IBGE] Instituto Brasileiro de Geografia e Estatística
	\item[IGP-DI] Índice Geral de Preços - Disponibilidade Interna
	\item[CMN] Conselho Monetário Nacional
	\item[ECM] Modelagem de Correção de Erros
	\item[VAR] Análise de Vetores Autoregressivos
	\item[PIB] Produto Interno Bruto
	\item[PAI] Programa de Ação Imediata
	\item[PEC] Proposta de Emenda à Constituição
	\item[BCB] Banco Central do Brasil
	\item[IPP] Índice de Preços ao Produtor
	\item[IPA-EP-DI] Índice de Preços ao Produtor Amplo, Estágios de Produção, Disponibilidade Interna
	\item[MQO] Método de Mínimos Quadrados Ordinários
	\item[MQG] Método de Mínimos Quadrados Generalizados
	\item[ADF] Dickey-Fuller Aumentado
\end{siglas}
\pagebreak


% %%%%%%%%%%%%%%%%
%LISTA DE SíMBOLOS
% \begin{simbolos}
% 	\item[Depeco] Departamento de Economia
% \end{simbolos}
% \pagebreak


% %%%%%%%%%%%%%%%%
% SUMÁRIO
\pdfbookmark[0]{\contentsname}{toc}
\tableofcontents*
\cleardoublepage


% %%%%%%%%%%%%%%%%
%INÍCIO DO TEXTO
\textual % indica o início do texto para a numeração das páginas
\pagestyle{simple}
\aliaspagestyle{chapter}{simple}

\chapter{Introdução}

O controle da inflação é um dos principais desafios enfrentados pelas autoridades econômicas em diversas nações ao redor do mundo. No Brasil, a adoção do Regime de Metas de Inflação (RMI) em 1999 representou uma mudança significativa na condução da política monetária, com o objetivo de estabilizar os preços e promover um ambiente econômico previsível \cite{fraga_2003_inflation}. Diante desse contexto, este trabalho tem como objetivo analisar a dinâmica inflacionária no Brasil sob o RMI entre os anos de 2003 e 2023, focando especificamente nos componentes de inflação de demanda e de oferta para verificar a efetividade das políticas do Banco Central.

Ao longo deste trabalho, serão abordados os diferentes tipos de inflação, seguido por um detalhamento do histórico econômico brasileiro e uma discussão sobre o RMI e a política monetária na terceira seção. A quarta seção detalhará a metodologia, os modelos utilizado e os dados coletados. Os resultados e a discussão serão apresentados na quinta seção, culminando com a conclusão na sexta seção. Por meio desta análise, espera-se contribuir para uma melhor compreensão da dinâmica inflacionária no Brasil e a eficácia do RMI na promoção da estabilidade econômica, assim como os impactos sobre o produto e o desenvolvimento econômico.

Para compreender o cenário a ser estudado, definimos o Regime de Metas de Inflação como um arranjo institucional em que o Banco Central se compromete a manter a inflação dentro de um intervalo preestabelecido, utilizando instrumentos de política monetária, como as operações em \textit{open market}, para alcançar esse objetivo \cite{svensson_1997_inflation}. Este regime busca ancorar as expectativas inflacionárias dos agentes econômicos, contribuindo para a estabilidade macroeconômica \cite{mishkin_2000_inflation}. Desde sua implementação, o RMI no Brasil tem se baseado em metas anuais de inflação definidas pelo Conselho Monetário Nacional (CMN), com o Banco Central ajustando a taxa do Sistema Especial de Liquidação e Custódia (SELIC) via instrumentos de política monetária para controlar a demanda agregada e manter a inflação dentro dos limites estipulados.

De acordo com o Banco Central do Brasil (2013), A taxa SELIC é obtida através do cálculo da taxa média ponderada e ajustada das operações de financiamento de um dia, que são garantidas por títulos públicos federais. Essas operações ocorrem no sistema financeiro ou em câmaras de compensação e liquidação de ativos, sob a forma de operações compromissadas, onde o vendedor se compromete a recomprar os títulos vendidos no dia útil seguinte. Abaixo, na equação 1.1, é dada a fórmula utilizada para o cálculo da taxa:

\begin{equation}
	\left[ \left( \left(\frac{\sum_{j=1}^{n} L_j \cdot V_j}{\sum_{j=1}^{n} V_j} \right)^{252} - 1 \right) \times 100 \right] \textrm{\% ao ano}
\end{equation}


A equação 1.1 baseia-se em três principais variáveis, sendo elas $L_j$, o fator diário correspondente à taxa da j-ésima operação, $V_j$, o valor financeiro correspondente à taxa da j-ésima operação e $n$, o número de operações que compõem a amostra. A taxa SELIC é principalmente ajustada pelas operações de mercado aberto (\textit{open market}), instrumento da política monetária onde o Banco Central realiza a compra ou venda de títulos da dívida pública aos principais bancos comerciais brasileiros.

Dessa forma, o ajuste da SELIC reflete os movimentos do Banco Central no controle da inflação ao influenciar a oferta de moeda e o custo do crédito no mercado. A variação na taxa básica de juros, promovida pelas operações de mercado aberto, tem, portanto, efeitos diretos sobre os níveis de consumo e investimento na economia. A política monetária, ao atuar na direção de modificar a taxa SELIC, busca equilibrar o crescimento econômico com as pressões inflacionárias, que podem ter diferentes origens e dinâmicas.

Assim, detalhamos os diferentes tipos de inflação experienciados pelo país para analisar como o Banco Central tem respondido às mudanças nas variáveis macroeconômicas. A economia brasileira, como muitas outras, está sujeita a um descompasso entre o crescimento econômico real e a base monetária, resultando em inflação. O processo inflacionário pode ter como fonte diversos fatores, sendo comum verificar a inflação de oferta, a inflação de demanda, a inflação inercial e a inflação estrutural. Cada um desses tipos de inflação apresenta características distintas que influenciam a eficácia do RMI.

Como exemplo, a inflação de oferta é impulsionada por choques nos custos de produção, como aumentos nos preços de matérias-primas ou salários \cite{blinder_2008_the}, porém aumentar a taxa de juros em resposta a esse tipo de inflação pode ser ineficaz uma vez que abre oportunidade para o resfriamento da economia, já que não combate a causa raiz do processo inflacionário. Compreender a interação entre os tipos de inflação e o funcionamento do RMI é crucial para avaliar a eficácia das políticas monetárias adotadas pelo Banco Central do Brasil nas últimas décadas. Nesse contexto, o presente trabalho pretende investigar como a inflação de demanda e de oferta influenciaram a dinâmica inflacionária no Brasil durante o período analisado e como o RMI respondeu a esses desafios.

Para isso, é, também, essencial utilizar técnicas econométricas aplicadas a séries temporais relacionadas aos índices de preços, produto interno e outros indicadores econômicos. A análise de séries temporais permite modelar e prever o comportamento de variáveis econômicas ao longo do tempo, capturando tanto as tendências quanto as flutuações cíclicas e sazonais \cite{enders_2015_applied}. Métodos como a Análise de Vetores Autoregressivos (VAR), a Modelagem de Correção de Erros (ECM) e os testes de raiz unitária são frequentemente utilizados para investigar a relação entre a política monetária e a inflação \cite{hamilton_2020_time}.

A aplicação de econometria a séries temporais envolve várias etapas, incluindo a identificação e a modelagem das propriedades estocásticas das séries de dados, a estimação dos parâmetros do modelo e a realização de testes de hipóteses para validar os resultados \cite{stock_2020_introduction}. Essas técnicas nos permite avaliar a resposta da inflação a choques de demanda e oferta, bem como a eficácia das intervenções do Banco Central no controle dos preços. A seguir, é feita a descrição aprofundada dos tipos de inflação que a economia brasileira observou historicamente e sua relação com o RMI.

\chapter{Tipos de Inflação}

Historicamente, o Brasil enfrentou diferentes tipos de inflação, cada um decorrente de fatores econômicos e contextos específicos. Nos anos de 1980 e início dos anos 1990, por exemplo, o país sofreu com altos níveis de inflação, caracterizado por aumentos de preços extremamente rápidos e incontroláveis. Esse período foi marcado por políticas econômicas instáveis, déficits fiscais elevados e um ciclo de indexação de preços, onde os preços e salários eram ajustados automaticamente à inflação passada, perpetuando o ciclo inflacionário. Para \citeonline{bresser_1990_hiperinfla}, o fator desencadeante da hiperinflação foi a crise fiscal dos anos 1980, que pode ser desmembrada em três elementos: o déficit público, a dívida pública, tanto interna quanto externa, e o curto prazo de vencimento dos títulos públicos.

% TODO: citation to back this up? https://www.scielo.br/j/rec/a/4ZqfqsrqsHwFJV9QjJWYsfN/?lang=pt

O Brasil experenciou de forma ostensiva a inflação inercial no final do século XX. Esse fenômeno é como um ciclo vicioso, uma vez que a inflação persiste devido à expectativa de que ela continuará, independentemente de outros fatores econômicos. Nos anos 1990, especialmente antes do Plano Real, a inflação inercial era alimentada pela indexação, onde preços e contratos eram ajustados automaticamente com base na inflação passada. O Plano Real, implementado em 1994, conseguiu quebrar essa inércia ao introduzir uma nova moeda e uma série de reformas econômicas que estabilizaram a economia. A partir desse ponto, o Brasil passou a experimentar uma inflação mais controlada, embora ainda enfrentasse desafios ocasionais devido a choques externos e flutuações internas na economia.

Abaixo, é apresentado um breve histórico das moedas que circularam na economia brasileira. É importante observar como a base monetária se expandiu ao longo do tempo, principalmente a partir do Cruzeiro na década de 1970, momento em que houve sucessivos "cortes de zero", onde a nova moeda possuia valores nas cédulas geralmente mil vezes menores do que a moeda anterior.

\vspace{0.2cm}

\begin{table}[H]
	\caption{Histórico de moedas brasileiras}
	\rowcolors{2}{gray!8}{white}
	\begin{tabular}{ | l || c | l | }
		\hline
		\textbf{Moeda Vigente}                      & \textbf{Período de Vigência} & \textbf{Equivalência}      \\
		\hline
		\textbf{Cruzeiro}                           & 1942 a 1964                  & Cr\$ 1,00 = Rs 1\$000      \\
		\textbf{Cruzeiro (retirada dos centavos)}   & 1964 a 1967                  & Cr\$ 1 = Cr\$ 1,00         \\
		\textbf{Cruzeiro Novo (volta dos centavos)} & 1967 a 1970                  & NCr\$ 1,00 = Cr\$ 1.000    \\
		\textbf{Cruzeiro}                           & 1970 a 1984                  & Cr\$ 1,00 = NCr\$ 1,00     \\
		\textbf{Cruzeiro (retirada dos centavos)}   & 1984 a 1986                  & Cr\$ 1 = Cr\$ 1,00         \\
		\textbf{Cruzado (volta dos centavos)}       & 1986 a 1989                  & Cz\$ 1,00 = Cr\$ 1.000     \\
		\textbf{Cruzado Novo}                       & 1989 a 1990                  & NCz\$ 1,00 = Cz\$ 1.000,00 \\
		\textbf{Cruzeiro}                           & 1990 a 1993                  & Cr\$ 1,00 = NCz\$ 1,00     \\
		\textbf{Cruzeiro Real}                      & 1993 a 1994                  & CR\$ 1,00 = Cr\$ 1.000,00  \\
		\textbf{Real}                               & Desde 1994                   & R\$ 1,00 = CR\$ 2.750,00   \\                                                                  
		\hline
	\end{tabular}
	\vspace{1ex}
	
	\noindent \footnotesize{Fonte: Histórico das Alterações da Moeda Nacional. Disponível em \\ https://web.archive.org/web/20151115205007/http://www.ocaixa.com.br/passos/passos2.htm.}
\end{table}
\vspace{-0.2cm}
\vspace{0.5cm}

Desde o Cruzeiro, passando pelo Cruzado, até chegar ao Real, cada uma das moedas surgiu em resposta a crises inflacionárias e à perda de confiança na moeda nacional. Essas mudanças foram frequentemente catalisadas por eventos externos, como crises internacionais, choques no preço do petróleo, e também por fatores internos, como descontrole fiscal e políticas econômicas inconsistentes. Abaixo, detalhamos como a instabilidade econômica que impacta diretamente a capacidade produtiva das empresas e o custo dos produtos e serviços contribui para a inflação de oferta.

\section{Inflação de Oferta}

Um exemplo marcante de inflação de oferta, tanto no Brasil quanto no mundo, ocorreu durante as crises do petróleo dos anos 1970. O aumento abrupto dos preços do petróleo, um insumo fundamental para diversas indústrias, resultou em custos mais altos de transporte e produção, impactando a economia brasileira. Outro exemplo mais recente foi observado durante a crise hídrica na região sudeste em 2014 e 2015, quando a escassez de água afetou a produção agrícola e a geração de energia hidrelétrica, levando ao aumento dos custos de alimentos e eletricidade (FRANCO, 2014).

É também conhecida como inflação de custos e ocorre quando os custos de produção aumentam, levando a um aumento nos preços dos bens e serviços finais. Esses aumentos nos custos podem ser decorrentes de elevações nos preços das matérias-primas, aumentos salariais, ou choques de oferta adversos, como desastres naturais ou interrupções no fornecimento de insumos essenciais. \citeonline{blinder_2008_the} descrevem a inflação de custos como um fenômeno onde os produtores, enfrentando maiores custos de produção, repassam esses custos para os consumidores através de preços mais altos.

"A inflação de custos, além de elevar o nível de preços da economia, causa estagflação, expressão empregada quando a economia apresenta alto nível de preços e contração do produto interno bruto, diferentemente da inflação de demanda, onde a expansão da demanda agregada, além de elevar o nível de preços, aquece a economia e estimula o crescimento do PIB". \cite{cortapasso_estagflacao}

Logo, combater a inflação de oferta envolve várias estratégias, incluindo a melhoria da infraestrutura para reduzir custos de produção e distribuição, a diversificação de fontes de insumos para diminuir a dependência de um único fornecedor e o incentivo à inovação e adoção de novas tecnologias para aumentar a produtividade. Além disso, políticas cambiais que estabilizem a moeda podem minimizar os impactos de variações cambiais nos custos de insumos importados. 

TODO: citation

Visto que o Banco Central do Brasil (BCB) não prioriza o canal cambial para a transmissão da política monetária (XYZ, 2020), ter o principal instrumento de política monetária concentrado nas operações de mercado aberto implica que, em um caso de inflação de custos, a resposta seria a modificação da taxa de juros básica da economia. Aumentar a taxa de juros em resposta a esse tipo de inflação, porém, pode ser ineficaz ou até prejudicial, pois, embora possa ajudar a conter os preços, também pode frear o crescimento econômico, agravando a situação ao desestimular investimentos e aumentar o custo do crédito. Esse tipo de medida tende a impactar mais fortemente a demanda, mas não resolve as causas subjacentes do aumento nos custos de produção.

\section{Inflação de Demanda}

Se por um lado choques nos custos de produção e salários causam inflação de oferta, a inflação de demanda ocorre quando a demanda agregada por bens e serviços supera a capacidade produtiva da economia, resultando em pressões inflacionárias. Este tipo de inflação é frequentemente associado a períodos de crescimento econômico robusto, onde a renda disponível e o consumo das famílias aumentam significativamente. Segundo \citeonline{olivierblanchard_2013_macroeconomics}, a inflação de demanda pode ser desencadeada por políticas fiscais expansionistas, como aumentos nos gastos governamentais ou cortes de impostos, que elevam a demanda agregada sem um correspondente aumento na oferta.

Um exemplo histórico significativo de inflação de demanda ocorreu durante a década de 1970, quando muitos países enfrentaram pressões inflacionárias devido a políticas fiscais expansionistas e aumentos rápidos nos gastos de consumo \cite{blinder_2008_the}. Segundo \cite{woodford_2009_interest}, a gestão eficaz da demanda agregada é crucial para evitar pressões inflacionárias, destacando a importância da coordenação entre políticas fiscais e monetárias. Portanto, o controle da inflação de demanda requer uma abordagem equilibrada que inclua a moderação das expansões fiscais e uma política monetária prudente que consiga antecipar e neutralizar excessos de demanda.

TODO: Falar mais sobre o RMI e seus impactos

TODO: 3rd phrase weird

Um dos principais efeitos observados sob a gestão macroeconômica brasileira é o aumento das taxas de juros, que desestimula o consumo e o investimento ao encarecer o crédito. Além disso, é possível realizar a redução dos gastos públicos e o aumento da tributação, ajudando a conter a demanda agregada, mas trazendo efeitos colaterais negativos que fazem com que essa metodologia não seja tão utilizada. Essas medidas visam equilibrar a demanda com a capacidade produtiva da economia, reduzindo a pressão sobre os preços.

\section{Inflação Estrutural}

Diferentemente das modalidades anteriores, a inflação estrutural é causada por desequilíbrios fundamentais na estrutura econômica de um país. Fatores como a rigidez dos mercados, a ineficiência produtiva e a falta de competitividade em determinados setores podem contribuir para esse tipo de inflação. A inflação estrutural é particularmente relevante em economias em desenvolvimento, onde a infraestrutura inadequada e a baixa produtividade agrícola podem levar a aumentos persistentes nos preços. Essa forma de inflação requer reformas estruturais profundas para melhorar a eficiência e a capacidade produtiva da economia.

No Brasil, a inflação estrutural pôde ser observada durante as décadas de 1970 e 1980, quando o país enfrentou uma inflação persistente devido a vários fatores estruturais na economia. Entre esses fatores estavam a ineficiência produtiva, a baixa competitividade das indústrias nacionais, a excessiva burocracia, e a rigidez do mercado de trabalho. Isso foi agravado pela indexação generalizada da economia, onde salários, contratos e preços eram ajustados automaticamente com base na inflação passada, criando um ciclo contínuo de reajustes que perpetuava a inflação.

Sob o contexto do RMI, a regulação excessiva da taxa de juros pode contribuir para um cenário de inflação estrutural, uma vez que condições adversas de crédito e disponibilidade de moeda podem impedir processos de crescimento, atualização tecnológica e inovação dos setores produtivos. Adiante, observaremos o reflexo da condução da política monetária sobre a economia brasileira a longo prazo e como tentativas de preservação da estabilidade monetária podem, contraditoriamente, desencadear processos de descontrole monetário.


TODO: better link

\section{Inflação Inercial}

% TODO: Como retratam Giambiagi et al. (2011), no Plano Collor II houve a tentativa de controle da inflação por meio do congelamento dos preços, visando acabar com a memória inflacionária, ou seja, desindexar a economia, contudo o plano reduziu a inflação somente no curto prazo, a solução veio com o Plano Real em 1994 que estabeleceu a Unidade Real de Valor (URV) para desindexar a economia e determinou o lançamento de uma nova unidade monetária que estaria indexada ao dólar, nascia assim o real.

Outra forma de inflação experenciada pelo Brasil na mesma época discutida na seção anterior foi a inflação inercial, que é associada à tendência dos índices de preços continuarem aumentando devido à persistência das expectativas inflacionárias passadas. Esse tipo de inflação pode ser causado pela indexação de preços e salários, onde os agentes econômicos ajustam automaticamente os preços e salários futuros com base na inflação passada. \citeonline{lopes_1985_inflao} explica que, em um ambiente de inflação inercial, as expectativas de inflação se tornam autorrealizáveis, perpetuando a continuidade da inflação mesmo na ausência de novos choques de demanda ou custos.

A indexação da economia, como já mencionada na seção 2.3 de inflação estrutural, fazia com que o ajuste automático dos salários e contratos com base na inflação passada criasse um ciclo contínuo de reajustes. Essa indexação perpetuava a inflação ao longo do tempo, principalmente nos anos de 1980, quando esse comportamento começou a ser identificado a partir do desenvolvimento de teorias a respeito (PEREIRA, 1998).

É notável que, apesar do plano Real ter sido um fator muito importante para limitar as tendências inflácionárias, inclusive de inflação	inercial, o crescimento dos índices de preços, como o IGP-DI visto na Figura \ref{fig:igpdiacum} foram superiores às metas estipuladas pelo Banco Central. Assim, os preços de bens e serviços subiram a um ritmo mais acelerado que o esperado e a população mais pobre acaba tendo seu orçamento familiar comprimido.

\begin{figure}[H]
	
	\caption{IGP-DI Acumulado x Meta de Inflação Acumulada (\%)}
	
	\includegraphics[]{fig/igpdi_96_24_t.png}\\
	\footnotesize \textbf{Fonte}: o próprio autor, dados fornecidos pelo Banco Central e Fundação Getúlio Vargas.
	
	\label{fig:igpdiacum}
\end{figure}

Na seção subsequente, observar-se-á que as principais fontes de inflação histórica no Brasil são associadas a políticas desenvolvimentistas e expansionistas. Nota-se que a inflação estrutural e a inflação inercial se sobressaltaram na economia brasileira, principalmente nas décadas de 1980 e 1990, até a execução bem-sucedida do Plano Real.


\chapter{Elementos Sobre Economia Brasileira}

Nessa seção, inicia-se a contextualização da história econômica do Brasil, assim como a análise do comportamento da política monetária ao longo do tempo. É feito, também, um detalhamento das principais características das políticas econômicas conduzidas por cada governo de 2003 a 2023 e seus impactos sobre o cenário econômico nacional.

\section{Governos do Século XX}

% TODO: O cenário econômico brasileiro entre a década de 80 e 90 foi marcado por anos de inflação recorrente, o regime militar deixara uma dívida externa estratosférica para os governos seguintes que tinham como principal meta o controle dos preços por meio de planos de estabilização. Sabe-se que todos os planos fracassaram na tentativa de conter o avanço da inflação, como explicam Moran e Witte (1993, p. 120): “No Brasil, são notórias as tentativas de conter este processo, a maioria não obteve o sucesso desejado por vários motivos. [...] dentre os quais destacam o Plano Cruzado I, o Plano Cruzado II, o Plano Bresser, o Plano Verão, o Plano Collor I e o Plano Collor II.”.

A bagagem econômica herdada dos diversos governos do século XX no Brasil teve um impacto duradouro na economia e na capacidade do país de implementar políticas macroeconômicas eficazes. Durante as décadas de 1950 e 1960, os governos de Getúlio Vargas e Juscelino Kubitschek adotaram políticas desenvolvimentistas que promoveram a industrialização acelerada e a expansão da infraestrutura, muitas vezes financiadas por meio de déficit público \cite{bielschowsky_2022_a}. Essas políticas resultaram em um crescimento econômico robusto, mas também em um aumento significativo da dívida pública e das pressões inflacionárias.

Sob o governo de Getúlio Vargas, a criação da Petrobras e da Companhia Siderúrgica Nacional (CSN) marcou um período de forte intervenção estatal na economia. O Estado Novo de Vargas (1937-1945) e seu segundo governo (1951-1954) buscaram consolidar a indústria de base no Brasil, visando reduzir a dependência de importações e fortalecer o mercado interno. Embora essas iniciativas tenham contribuído para a modernização da economia brasileira, elas também levaram a um aumento dos gastos públicos e ao crescimento da dívida externa \cite{fabiogiambiagi_2016_economia}.

Durante o governo de Juscelino Kubitschek (1956-1961), a política econômica se concentrou no Plano de Metas, que buscava ``50 anos em 5'' de desenvolvimento. Esse plano incluiu a construção de Brasília, a nova capital federal, e a expansão da infraestrutura de transporte e energia. Embora esses projetos tenham impulsionado a industrialização e o crescimento econômico, eles também resultaram em um aumento significativo do déficit público e da inflação \cite{bielschowsky_2022_a}. A rápida expansão foi financiada por meio de empréstimos externos, deixando o país vulnerável a crises de balanço de pagamentos.

Nos anos de 1960, a instabilidade política e econômica levou ao golpe militar de 1964, que instaurou um regime autoritário com uma nova agenda econômica. O governo militar inicialmente adotou medidas de austeridade para controlar a inflação e estabilizar a economia. Contudo, o período mais notável de crescimento, conhecido como o "milagre econômico brasileiro" (1968-1973), foi caracterizado por investimentos massivos em infraestrutura e pela abertura econômica. Esse crescimento foi impulsionado por políticas fiscais e monetárias expansivas, que, embora tenham gerado um aumento substancial do produto interno bruto (PIB), também ampliaram a dívida externa e interna \cite{amaurypatrickgremaud_2009_economia}.

O ``milagre econômico'' foi marcado por grandes projetos, como a construção da Rodovia Transamazônica e a usina de Itaipu, que demandaram enormes recursos financeiros. Embora esses projetos tenham impulsionado a economia, eles também aumentaram as pressões inflacionárias. Além disso, a concentração de renda e a repressão aos movimentos sociais geraram tensões sociais significativas. O choque do petróleo de 1973 exacerbou as vulnerabilidades econômicas, pois aumentou os custos de importação de energia e pressionou ainda mais a balança de pagamentos.

Isso fez com que a década de 1980, chamada de ``década perdida'', fosse marcada por estagnação econômica, hiperinflação e crise da dívida externa. A política econômica dos governos militares posteriores, particularmente sob a presidência de João Figueiredo (1979-1985), foi incapaz de lidar com as consequências dos choques externos e das políticas expansionistas anteriores. A moratória da dívida em 1987 simbolizou a profundidade da crise econômica \cite{fabiogiambiagi_2016_economia} na época. As tentativas de estabilização, como o Plano Cruzado (1986) durante o governo de José Sarney, falharam em resolver os problemas estruturais e levaram a surtos de hiperinflação.

O Plano Cruzado tentou controlar a inflação por meio do congelamento de preços e salários e da introdução de uma nova moeda, o cruzado. Inicialmente, o plano teve sucesso em reduzir a inflação, mas a falta de controle sobre os gastos públicos e a resistência política levaram ao seu fracasso. A inflação retornou, exacerbada pela falta de ajustes estruturais e pela deterioração das contas públicas \cite{fabiogiambiagi_1999_a}.

A transição para a democracia em 1985 trouxe novos desafios econômicos. O governo de Fernando Collor (1990-1992) implementou o Plano Collor, que incluía o confisco de depósitos bancários e a tentativa de liberalização econômica. Embora tenha reduzido temporariamente a inflação, o plano causou uma grave recessão econômica e enfrentou forte oposição política, resultando em sua rápida deterioração e na posterior hiperinflação \cite{lacerda_2010_economia}.

Durante o governo de Itamar Franco (1992-1995), o Brasil finalmente começou a estabilizar sua economia com o Plano Real, liderado pelo então Ministro da Fazenda Fernando Henrique Cardoso. O plano introduziu uma nova moeda, o real, e implementou âncoras cambiais e políticas fiscais rígidas. Essas medidas conseguiram estabilizar a inflação e criar uma base para o crescimento econômico sustentável. O sucesso do Plano Real foi um ponto de inflexão na história econômica do Brasil, marcando o fim de um ciclo de políticas populistas e instabilidade econômica \cite{lacerda_2010_economia}.

A instabilidade econômica e a subsequente estabilização pela qual o Brasil passou no período anterior e posterior ao Plano Real é claramente observada nas variações do Índice Geral de Preços - Disponibilidade Interna (IGP-DI). A Figura \ref{fig:igpdi}, a seguir, apresenta a série temporal do comportamento da inflação entre 1985 e 1996 no Brasil, com destaque para as diversas tentativas de estabilização realizadas.

\begin{figure}[H]
	
	\caption{Comportamento da Inflação Mensal - IGP-DI - 1985-1996 (\%)}
	
	\includegraphics[]{fig/igp-di.png}\\
	
	\footnotesize Fonte: FGV.
	
	\label{fig:igpdi}
	
\end{figure}

Com destaque, a inflação no Brasil (Figura \ref{fig:igpdi}), observou diversos choques resultantes de políticas ineficazes no controle dos preços. O legado dos governos populistas do século XX, com suas políticas desenvolvimentistas e intervenções estatais, deixou uma economia caracterizada por altas taxas de inflação, dívida pública crescente e instabilidade macroeconômica. A implementação do RMI em 1999 representou um esforço para romper com esse passado e adotar uma política monetária mais disciplinada e previsível, visando garantir a estabilidade de preços e promover o crescimento sustentável a longo prazo \cite{amaurypatrickgremaud_2009_economia}.

\section{Transição para a Estabilidade: Plano Real e a Introdução do RMI}

O Plano Real, implementado em 1994 durante o governo de Itamar Franco e sob a liderança do então Ministro da Fazenda Fernando Henrique Cardoso, marcou um ponto de inflexão na luta contra a inflação no Brasil. O plano envolveu uma série de medidas macroeconômicas, incluindo a criação de uma nova moeda, o real, a introdução de âncoras cambiais e a implementação de políticas fiscais rigorosas \cite{amaurypatrickgremaud_2009_economia}. Essas medidas conseguiram estabilizar a inflação e restaurar a confiança na economia brasileira.

A estabilidade alcançada pelo Plano Real criou as condições necessárias para a adoção do Regime de Metas de Inflação (RMI) em 1999. Esse novo regime, que enfatiza a transparência e a credibilidade das políticas monetárias, representou uma mudança de paradigma na abordagem do Brasil ao controle da inflação. Ao focar em metas de inflação como guia principal para a política monetária, o Banco Central do Brasil conseguiu ancorar as expectativas inflacionárias e promover um ambiente macroeconômico mais estável.

A implementação do Plano Real começou com uma série de medidas preparatórias, conhecidas como o Programa de Ação Imediata (PAI), que visavam reduzir a inércia inflacionária e estabilizar a economia antes da introdução da nova moeda. Essas medidas incluíram o ajuste fiscal, a redução do déficit público e a liberalização do comércio. Em julho de 1994, o real foi finalmente introduzido, substituindo o cruzeiro real a uma taxa de paridade inicial de 1:1 com o dólar norte-americano. A âncora cambial foi uma ferramenta crucial para estabilizar as expectativas inflacionárias e fortalecer a credibilidade da nova moeda \cite{silva_2002_plano}. 

A introdução do real foi acompanhada por uma política monetária restritiva, com altas taxas de juros para controlar a demanda agregada e evitar pressões inflacionárias. O governo também adotou um regime de câmbio fixo, permitindo que o Banco Central interviesse para manter a paridade da moeda. Essas medidas foram fundamentais para o sucesso inicial do Plano Real, que conseguiu reduzir a inflação anual de mais de 2.000\% em 1993 para menos de 10\% em 1996 \cite{fabiogiambiagi_1999_a}.

O sucesso do Plano Real não foi isento de desafios. A âncora cambial, enquanto eficaz no curto prazo, gerou pressões sobre a balança de pagamentos do Brasil, levando a um aumento do déficit em conta corrente. Além disso, a apreciação do real tornou as exportações brasileiras menos competitivas, aumentando a dependência de capital externo para financiar o déficit. Esses desequilíbrios se tornaram evidentes durante as crises financeiras internacionais da segunda metade da década de 1990, como a crise do México (1994-1995), a crise asiática (1997) e a crise da Rússia (1998) \cite{santna_2002_crises}.

Em resposta a essas crises e às pressões cambiais crescentes, o governo brasileiro decidiu abandonar a âncora cambial em janeiro de 1999 e adotar um regime de câmbio flutuante. Essa mudança foi acompanhada pela introdução do Regime de Metas de Inflação (RMI), que visava ancorar as expectativas inflacionárias por meio da transparência e previsibilidade das ações do Banco Central. Sob o RMI, o Conselho Monetário Nacional estabelece metas de inflação anuais, e o Banco Central utiliza as operações de \textit{open market}, redesconto e depósito compulsório como  principais instrumentos para atingir essas metas.

% TODO: No segundo mandato de FHC, entre 1999 e 2002, o país precisou recorrer ao Fundo Monetário Internacional (FMI) para enfrentar o problema externo, recebendo acote de ajuda em torno de U$ 40 bilhões. O acordo com o FMI enfrentou duas adversidades, Giambiagi et al. (2011) explicam que a desconfiança referente a não desvalorização do câmbio por parte do mercado e o desprezo pelo Congresso com relação a medida de contribuição previdenciária, imposta pelo plano, levou a rejeição da medida pelos congressistas e aumentou o pessimismo, acelerando o processo de perda de divisas.

A transição para o RMI representou uma mudança significativa na política monetária do Brasil. Ao invés de se concentrar exclusivamente no controle do câmbio, o Banco Central passou a focar na estabilidade de preços como seu principal objetivo. Essa abordagem permitiu uma maior flexibilidade na condução da política monetária, facilitando a resposta a choques econômicos internos e externos. A transparência do regime de metas também aumentou a credibilidade das políticas do Banco Central, ajudando a ancorar as expectativas inflacionárias e a reduzir a inflação de forma sustentável.

Nos primeiros anos do RMI, a economia brasileira enfrentou desafios significativos, incluindo uma crise cambial em 1999 e um cenário internacional adverso. No entanto, a política monetária firme e a disciplina fiscal contribuíram para a estabilização da economia. A inflação, que chegou a 8,9\% em 1999, foi gradualmente reduzida, atingindo 5,97\% em 2002. A credibilidade do Banco Central e a eficácia do RMI foram reforçadas pela consistência das políticas adotadas, mesmo diante de adversidades \cite{fabiogiambiagi_1999_a}.

Os governos de Fernando Henrique Cardoso (1995-2002) implementaram reformas estruturais importantes que complementaram a estabilidade macroeconômica promovida pelo Plano Real e pelo RMI. Entre essas reformas estavam a privatização de empresas estatais, a liberalização do mercado financeiro e a promoção de ajustes fiscais. Essas medidas contribuíram para aumentar a eficiência econômica e trazer ao Brasil características mais semelhantes às de países desenvolvidos \cite{fabiogiambiagi_2016_economia}.

No início dos anos 2000, o Brasil começou a colher os frutos das reformas e da estabilidade macroeconômica. A economia cresceu, a inflação permaneceu sob controle e a dívida pública passou a ser mais controlada. Apesar de parte do crescimento ter sido eclipsado pela crise energética de 2001 \cite{fabiogiambiagi_2016_economia}, as reformas monetárias foram um sucesso.

Em resumo, a transição para a estabilidade econômica no Brasil foi marcada pela implementação bem-sucedida do Plano Real e a adoção do Regime de Metas de Inflação. Essas mudanças representaram um rompimento com o passado de instabilidade e políticas populistas, estabelecendo as bases para um crescimento econômico sustentável e uma maior credibilidade das instituições econômicas brasileiras. A disciplina fiscal, a independência do Banco Central e a transparência das políticas monetárias foram elementos-chave para o sucesso dessa transição, que continua a influenciar positivamente a economia brasileira até os dias de hoje.

A partir de 2003, a economia brasileira passou por diversas mudanças econômicas e políticas que influenciaram a condução e os resultados do RMI. Os períodos analisados a seguir foram marcados por diferentes administrações que implementaram políticas econômicas distintas, refletindo na dinâmica inflacionária e na eficácia do regime. A seguir, é detalhada a evolução do quadro econômico brasileiro ao longo das diversas gestões.

\section{Período Lula (2003-2010)}

Durante o governo de Luiz Inácio Lula da Silva (2003-2010), o Brasil experimentou um período de forte crescimento econômico, impulsionado pela alta nos preços das \textit{commodities} e um aumento significativo nos investimentos estrangeiros diretos. A gestão do Banco Central, sob a liderança de Henrique Meirelles, manteve uma política monetária rigorosa para controlar a inflação. A taxa SELIC foi usada de forma ativa para garantir que a inflação permanecesse dentro das metas estabelecidas pelo Conselho Monetário Nacional (CMN). Segundo \citeonline{silva_2016_politica}, a política fiscal contracionista adotada nos primeiros anos do governo Lula contribuiu para a credibilidade dos mandatos, permitindo uma redução gradual da taxa de juros sem comprometer a estabilidade dos preços.

O período também foi marcado por um aumento na credibilidade internacional do Brasil, com a economia apresentando crescimento robusto e uma significativa redução na vulnerabilidade externa. O superávit primário foi mantido, garantindo um controle mais rígido das contas públicas e contribuindo para a estabilidade macroeconômica. A combinação de uma política fiscal responsável com uma política monetária eficiente permitiu que o Brasil navegasse com sucesso por um cenário econômico global inicialmente favorável, mas que se tornou desafiador com a crise financeira de 2008 \cite{fabiogiambiagi_2016_economia}.

A crise financeira global de 2008 representou um grande desafio para a economia brasileira. O governo Lula, em resposta, adotou uma série de medidas para mitigar os impactos da crise, incluindo a redução da taxa SELIC e a implementação de pacotes de estímulo fiscal para sustentar o consumo e o investimento. Essas medidas, associadas à robustez do sistema financeiro brasileiro e à atuação prudente do Banco Central, permitiram que o Brasil enfrentasse a crise de maneira mais resiliente em comparação com muitas outras economias emergentes e desenvolvidas.

Além das políticas de estabilização econômica, o governo Lula implementou programas sociais de grande alcance, como o Bolsa Família, que contribuíram para a redução da pobreza e da desigualdade social. Esses programas tiveram um efeito positivo no aumento da demanda interna, apoiando o crescimento econômico. A inclusão social promovida por esses programas não apenas melhorou as condições de vida de milhões de brasileiros, mas também criou um mercado consumidor interno mais dinâmico, o que foi fundamental para o crescimento sustentado da economia durante esse período.

No entanto, o aumento do consumo e a expansão dos programas sociais também colocaram pressão sobre a inflação. A execução do RMI levantou críticas sobre os altos custos dos empréstimos e o impacto negativo sobre o investimento e o crescimento econômico a longo prazo. Dessa forma, o regime de metas de inflação trouxe impactos para a economia brasileira durante o governo Lula, proporcionando estabilidade de preços, mas também impondo restrições à flexibilidade econômica.

TODO: passar citações para refs.bib

\section{Período Dilma Rousseff (2011-2016)}
O governo de Dilma Rousseff (2011-2016) enfrentou desafios econômicos significativos, incluindo uma desaceleração econômica e aumento das pressões inflacionárias. A política econômica adotada durante este período foi caracterizada por uma expansão fiscal e intervenções frequentes nos preços administrados, como combustíveis e energia, o que dificultou o controle da inflação. A credibilidade do governo foi colocada em questão devido ao uso de medidas não convencionais e à percepção de interferência política no Banco Central. Como resultado, a inflação frequentemente excedeu as metas estabelecidas, exigindo aumentos abruptos na taxa de juros para tentar trazer a inflação de volta ao intervalo meta \cite{carvalho_2016_growth}.

A implementação de políticas expansionistas, conhecidas como "nova matriz econômica", buscava incentivar o crescimento econômico através do aumento dos gastos públicos e subsídios a setores estratégicos. No entanto, essas políticas resultaram em um aumento do déficit fiscal e da dívida pública, o que elevou as expectativas inflacionárias. As desonerações tributárias e a política de controle de preços contribuíram para a deterioração das contas públicas e limitaram a capacidade do Banco Central de utilizar a política monetária para controlar a inflação de maneira eficaz.

O contexto internacional também influenciou significativamente a economia brasileira durante o governo Dilma. A queda nos preços das \textit{commodities} e a desaceleração do crescimento global impactaram negativamente a balança comercial brasileira, reduzindo as receitas de exportação e aumentando a vulnerabilidade externa do país. Esses fatores externos, aliados às políticas domésticas expansionistas, intensificaram as pressões inflacionárias e tornaram o ambiente econômico mais desafiador para a manutenção da estabilidade dos preços.

A falta de coordenação entre as políticas fiscal e monetária foi um dos principais problemas enfrentados durante esse período. Enquanto o Banco Central tentava controlar a inflação através do aumento da taxa SELIC, a política fiscal expansionista do governo gerava pressões contrárias, dificultando o alcance das metas inflacionárias. Esse descompasso entre as políticas minou a eficácia do RMI e aumentou a incerteza econômica, afetando negativamente a confiança dos investidores e consumidores.

Ao final do segundo mandato de Dilma Rousseff, a economia brasileira enfrentava uma grave recessão, com inflação elevada e desemprego em alta. A crise política e econômica culminou no \textit{impeachment} da presidente em 2016. A gestão subsequente herdou um cenário econômico complexo, marcado pela necessidade urgente de ajustes fiscais e de recuperação da credibilidade. Esse período evidenciou a importância de uma coordenação eficaz entre as políticas fiscal e monetária para a manutenção da estabilidade econômica e do controle inflacionário.

\section{Período Michel Temer (2016-2018)}

Com a chegada de Michel Temer à presidência em 2016, houve uma mudança significativa na política econômica do Brasil. Após o \textit{impeachment} de Dilma Rousseff, o governo Temer buscou implementar uma série de reformas estruturais com o objetivo de restaurar a confiança dos mercados e estabilizar a economia brasileira. Entre as principais medidas adotadas estavam a reforma trabalhista e a  proposta de emenda à constituição (PEC) do teto dos gastos, que limitou o crescimento dos gastos públicos à taxa de inflação do ano anterior por vinte anos. Essas reformas foram vistas como cruciais para o ajuste fiscal e para a recuperação da credibilidade econômica do país.

A gestão do Banco Central sob a liderança de Ilan Goldfajn foi marcada por um compromisso firme com o controle da inflação e a estabilização macroeconômica. Uma das primeiras ações de Goldfajn foi ancorar as expectativas inflacionárias, o que permitiu uma redução gradual e consistente da taxa SELIC ao longo de seu mandato. Essa abordagem contribuiu significativamente para a redução da inflação, que voltou a se situar dentro das metas estabelecidas pelo Conselho Monetário Nacional (CMN) após os elevados índices registrados no final do governo Dilma Rousseff.

Além das reformas estruturais e da política monetária rigorosa, o governo Temer também enfrentou o desafio de recuperar a confiança dos investidores e reverter a recessão econômica. O impacto positivo dessas medidas começou a ser percebido já em 2017, quando o Brasil registrou um leve crescimento econômico após dois anos de recessão profunda. A combinação de um ambiente regulatório mais favorável e de uma política monetária estável ajudou a atrair investimentos e a estimular a recuperação do emprego e da renda.

No entanto, o governo Temer também enfrentou dificuldades, incluindo a resistência política às reformas e os escândalos de corrupção que marcaram seu mandato. Apesar das controvérsias, a continuidade da política econômica voltada para a austeridade fiscal e a estabilidade monetária foi fundamental para consolidar os avanços obtidos na estabilização da economia.

\section{Período Jair Bolsonaro (2019-2022)}

O governo de Jair Bolsonaro (2019-2022) enfrentou desafios econômicos significativos, incluindo a pandemia de COVID-19, que impactou profundamente a economia global e doméstica. A resposta inicial do governo incluiu medidas fiscais expansivas para mitigar os efeitos econômicos da pandemia, como o auxílio emergencial e programas de manutenção de emprego, o que aumentou as preocupações inflacionárias. Essas medidas, apesar de essenciais para a proteção social e a manutenção da renda, pressionaram as contas públicas e elevaram a relação dívida/PIB do país.

Durante esse período, Roberto Campos Neto, presidente do Banco Central, manteve uma postura vigilante, ajustando a taxa SELIC conforme necessário para conter as pressões inflacionárias. Inicialmente, a taxa de juros foi reduzida para níveis historicamente baixos a fim de estimular a economia (2\%), mas a partir de 2021, devido ao aumento das pressões inflacionárias, houve um ciclo de elevação da SELIC para controlar a inflação, gerando uma alta significativa (13,75\%).

A pandemia também causou disrupções significativas nas cadeias de suprimentos globais, resultando em aumentos nos preços dos alimentos, energia e outros bens essenciais. Esses choques de oferta, combinados com uma depreciação cambial significativa, contribuíram para o aumento da inflação no Brasil. Apesar dessas dificuldades, o Banco Central adotou medidas de política monetária para evitar que a inflação se descontrolasse, influenciando proativamente a taxa de juros.

A política fiscal do governo Bolsonaro, particularmente durante a pandemia, foi caracterizada por um aumento substancial nos gastos públicos para enfrentar a crise sanitária e econômica. No entanto, essa expansão fiscal foi acompanhada de uma preocupação crescente com a sustentabilidade das contas públicas. A combinação de um elevado déficit fiscal e a crescente dívida pública gerou incertezas sobre a capacidade do governo de cumprir as metas fiscais e de manter a estabilidade econômica a longo prazo.

Além dos desafios econômicos, o governo Bolsonaro também enfrentou uma série de crises políticas que impactaram a economia. A relação tensa entre o poder executivo e os outros poderes, além das constantes mudanças nos ministérios, contribuiu para um ambiente de incerteza. No entanto, a independência formal do Banco Central, conquistada em 2021, foi vista como um passo importante para fortalecer o RMI e garantir uma política monetária mais eficiente e menos sujeita a pressões políticas, contribuindo para a estabilidade macroeconômica.

\section{Perspectivas Recentes e Desafios (2023)}

À medida que o Brasil entra em 2023, enfrenta uma série de desafios e oportunidades econômicas. Um dos principais focos continua sendo o controle da inflação, que permanece uma preocupação central para o Banco Central. Sob a liderança de Roberto Campos Neto, a política monetária tem sido rigorosamente ajustada para garantir que a inflação fique dentro das metas estabelecidas, apesar das pressões inflacionárias globais e domésticas.

\begin{figure}[H]
	
	\caption{Variação do IPCA x Meta de Inflação (BCB) (\%)}
	
	\includegraphics[]{fig/ibge_ipca_bcb_meta_99_23_t.png}\\
	
	\footnotesize \textbf{Fonte}: o próprio autor, dados fornecidos pelo Banco Central e IBGE.
	
\end{figure}

TODO: incluir mais um gráfico sintetizando PIB, inflação etc.

% \section{}

% A alta taxa de juros necessária para manter a inflação sob controle tem um impacto negativo sobre o crescimento econômico e o emprego. Os custos de empréstimos elevados desincentivam investimentos e consumo, limitando o crescimento do PIB

% A pressão para atingir metas anuais pode levar a uma política monetária excessivamente restritiva ou expansiva, prejudicando a estabilidade econômica de longo prazo. A tentativa de atingir metas de curto prazo pode negligenciar problemas estruturais mais profundos na economia brasileira

% Dominancia fiscal?


%%%%%%%%%%%%%
%BIBLIOGRAFIA
\bibliographystyle{abntex2-alf}
\bibliography{ref}



%%%%%%%%%%%%%%%%%
%APÊNCICES
% \apendices

% \chapter{Anexos e Apêndices, quando for o caso.}

% \chapter{Cronograma}

% \begin{table}[H]
% 	\centering
% 	\caption{Cronograma}
% 	\begin{tabular}{|c|c|c|c|c|c|c|c|c|c|c|c|}
% 		\toprule
% 		Atividades\textbackslash{}Data & Jan & Fev & Mar & Abr & Mai & Jun & Jul & Ago & Set & Out & Nov \\
% 		\midrule
% 		Atividade 1                    &     &     &     &     &     &     &     &     &     &     &     \\
% 		\midrule
% 		Atividade 2                    &     &     &     &     &     &     &     &     &     &     &     \\
% 		\midrule
% 		                               &     &     &     &     &     &     &     &     &     &     &     \\
% 		\midrule
% 		                               &     &     &     &     &     &     &     &     &     &     &     \\
% 		\midrule
% 		                               &     &     &     &     &     &     &     &     &     &     &     \\
% 		\midrule
% 		                               &     &     &     &     &     &     &     &     &     &     &     \\
% 		\midrule
% 		                               &     &     &     &     &     &     &     &     &     &     &     \\
% 		\midrule
% 		                               &     &     &     &     &     &     &     &     &     &     &     \\
% 		\midrule
% 		                               &     &     &     &     &     &     &     &     &     &     &     \\
% 		\midrule
% 		                               &     &     &     &     &     &     &     &     &     &     &     \\
% 		\midrule
% 		                               &     &     &     &     &     &     &     &     &     &     &     \\
% 		\midrule
% 		                               &     &     &     &     &     &     &     &     &     &     &     \\
% 		\bottomrule
% 	\end{tabular}%
% 	\label{tab:addlabel}%
% \end{table}%


% \chapter{Dicas}

% \section {Figuras e Tabelas}

% \subsection{Figuras}

% Para adicionar uma figura, crie o ambiente de figura e utilize \verb|\includegraphics| para anexar a figura no texto. Com esse comando você pode definir a largura e altura da figura no texto. 

% Exemplo: 

% \vspace{0.2cm}

% \verb|\begin{figure}[H]|

% \verb|\caption{Logo da UEL}|

% \verb|\includegraphics[width=7cm,height=2cm]{Logo_Uel}\\|

% \verb|{\footnotesize Fonte: ... }|

% \verb|\end{figure}|

% \vspace{0.5cm}

% Compilando esses comandos chegamos ao seguinte resultado: 


% \begin{figure}[H]
% 	\caption{Logo da UEL}
% 	\includegraphics[width=7cm,height=2cm]{Logo_Uel}\\
% 	{\footnotesize Fonte: ... }
% \end{figure}

% {\bfseries Importante:} As figuras devem estar na mesma pasta que o TCC está salvo.	

% \subsection{Tabelas}

% Escrever tabelas no {\LaTeX} sempre foi um ato de coragem dado o tempo necessário para completar a tarefa. 

% Para escrever uma tabela você precisa primeiro criar o ambiente tabular, definir quantas colunas indicando seu posicionamento (\verb|l = esquerda; c=centro; r=direita|) e adicionar os dados separados por \verb|&| e no final de cada linha indicar que ela terminou escrevendo \verb|\\| (quebra de linha). Linhas verticais podem ser geradas adicionando o caracter | e linhas horizontais o comando \verb|\hline|.

% \vspace{0.2cm}

% Exemplo:

% \verb|\begin{table}[H]|

% \verb|\caption{Nossa Tabela}|

% \verb|\begin{tabular}{ c c }|

% \verb|\hline|

% \verb|1 & 2 \\|

% \verb|4 & 5 \\|

% \verb|\hline|

% \verb|\end{tabular}\\|

% \verb|{\footnotesize Nota: ... }|

% \verb|\end{table}|

% \vspace{0.2cm}

% Compilando esses comando chegamos ao seguinte resultado: 

% \vspace{0.2cm}

% \begin{table}[H]
% 	\caption{Nossa Tabela}
% 	\begin{tabular}{ c  c }
% 		\hline
% 		1 & 2 \\
% 		4 & 5 \\
% 		7 & 8 \\
% 		\hline
% 	\end{tabular}\\
% 	{\footnotesize Nota: ... }
% \end{table}
% \vspace{0.5cm}

% Entretanto uma maneira mais fácil é utilizar um conversor.  No nosso caso vamos utilizar um conversor de EXCEL par {\LaTeX}. Baixe o conversor Excel2LaTeX no seguinte link: {$\href{https://ctan.org/tex-archive/support/excel2latex?lang=en}{https://ctan.org/tex-archive/support/excel2latex?lang=en}$} e adicione o conversor aos \textit{Add-ins} do Excel. 


% \section{Ambiente Matemático}

% Quando estamos no meio do texto podemos criar um ambiente matemático utilizando cifrões. Por exemplo, \verb|$Y_t=K_t^\alpha L_t^{1-\alpha}$| sairá no texto como $Y_t=K_t^\alpha L_t^{1-\alpha}$. Observe que os caracteres \verb|_| gera subscrito e \verb|^| gera o sobrescrito. Entretanto, para escrevermos equações que mereçam destaque, devemos criar o ambiente equation, o qual numerará automaticamente as equações.  

% \vspace{0.2cm}

% Exemplo:

% \vspace{0.2cm}

% \verb|\begin{equation}|

% \verb|Y_t=K_t^\alpha L_t^{1-\alpha}|

% \verb|\end{equation}|

% \vspace{0.2cm}

% Compilando esses comando chegamos a seguinte equação

% \vspace{0.2cm}

% \begin{equation}\label{equ1}
% 	Y_t=K_t^\alpha L_t^{1-\alpha}
% \end{equation}

% Algumas operações que normalmente usamos são frações, somatórias, limites, derivas e integrais:

% \begin{itemize}
% 	\item Fração = \verb|\frac{numredaor}{denominador}| = $\frac{Numerador}{Denominador}$

% 	\item Somatória = \verb|\sum_{mínimo}^{máximo} | = $\sum_{min}^{max}$

% 	\item Derivada = \verb|\partial| = $\partial$

% 	\item Integral, \verb|\int_{mínimo}^{máximo}| = $\int_{min}^{max}$

% \end{itemize}

% No caso das matrizes precisamos criar o ambiente de matrizes, bmatrix e escreve-la de forma parecida como escrevemos uma tabela.

% \vspace{0.2cm}
% Exemplo: 
% \vspace{0.2cm}

% \verb|\[|

% M=

% \verb|\begin{bmatrix}|

% \verb|1 & 2  \\|

% \verb|3 & 4 |

% \verb|\end{bmatrix}|

% \verb|\]|

% \vspace{0.4cm}

% Nossa Matriz:

% \vspace{0.2cm}
% \[
% 	M=
% 	\begin{bmatrix}
% 		1 & 2   \\
% 		3 & 4 
% 	\end{bmatrix}
% \]




% \section{Citações e Referências}\label{citaebib}

% {\tiny }

% Para fazer citações e a seção de referências é necessário utilizar um pacote de apoio as várias opções. No nosso caso estamos utilizando o pacote \verb|\usepackage[alf]{abntex2cite}|, o qual traz as normas da ABNT.

% \subsection{Arquivo .bib}

% Para gerar sua bibliografia é necessário criar um aquivo que contêm a sintaxe padrão da extensão .bib das referências que serão utilizadas. Cada referência deve apresentar uma lista de informações que serão utilizadas na geração das referências. 

% O exemplo abaixo mostra a entrada no aquivo .bib de duas teses de doutorado. Cada entrada é definida como uma tese, @phdthesis, que será chamada pelo {\LaTeX} como alexopoulos2012three e a outra bego2017three. Nas referências serão utilizadas as informações do título, nome do autor, ano e instituição.  

% \vspace{1cm}
% {
% \noindent
% @phdthesis{alexopoulos2012three,\\
% title={Three essays on inequality},\\
% author={Alexopoulos, Joanna},\\
% year={2012},\\
% school={University of Illinois at Urbana-Champaign}\\
% }
% }

% \vspace{1cm}
% {
% \noindent
% @phdthesis{bego2017three,\\
% title={Three essays on agricultural markets},\\
% author={Bego, Marcelo da Silva},\\
% year={2017}\\
% }
% }


% Você não precisa gerar a entrada de cada artigo, pois o Google Scholar oferece esse formato. Em cite no Google Scholar procure por BibTeX. 

% Nomeando e salvando o aquivo que contêm as referências de ref.bib (ou qualquer nome que você deseja) na mesma pasta que o TCC está salvo é possível adicionar a citação no texto e o \LaTeX \hspace{0.1cm} gerará a referência automaticamente. 

% 	{{\bfseries Importante:} Para compilar o aquivo ref.bib use F8}. Assim use F8 para compilar a entrada de novas referências e F5  para compilar o texto com a nova entrada de referência e atualizar o pdf.  

% \subsection{Citações}

% Para fazer a citação com \verb|\usepackage[alf]{abntex2cite}| temos as seguintes opções: \verb|\citet|,  \verb|\citeonline|.

% Exemplos: 

% \verb|\cite{alexopoulos2012three}| = (ALEXOPOULOS, 2012)

% \verb|\citeonline{alexopoulos2012three}| = Alexopoulos (2012)

% \verb|\cite{bego2017three}| = (BEGO, 2017)

% \verb|\citeonline{bego2017three}| = Bego (2017)

% \subsection{Referências}

% A bibliografia é gerada e atualizada automaticamente cada vez que adicionamos ou retiramos uma citação no corpo do texto. Para gerar a seção referências utiliza-se o seguintes comandos:  

% \vspace{1cm}
% {
% 	\noindent
% 	\verb|\bibliographystyle{abntex2-alf}%Definição do padrão ANBT| \\
% 	\verb|\bibliography{ref}%chamando o arquivo .bib com as referências | \\
% }

% Como resultado as referências bibliográficas do texto serão apresentadas da seguinte forma: 

% \vspace{1cm}
% \begin{center}
% 	\large \bfseries REFERÊNCIAS
% \end{center}

% {
% \noindent
% ALEXOPOULOS, J. Three essays on inequality. Tese (Doutorado)—University of Illinois at
% Urbana-Champaign, 2012.

% \noindent
% BEGO, M. d. S. Three essays on agricultural markets. Tese (Doutorado), 2017.
% }


\end{document}
